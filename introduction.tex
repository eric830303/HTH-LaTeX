\section{INTRODUCTION}
Due to escalating complexity of hardware design and manufacturing~\cite{wilson2013international}, integrated circuits (ICs) are usually designed and fabricated in multiple locations worldwide. Moreover, some design tools are also supplied from different units. With the utilization of third-party IP components and off-shore IC manufacturing, the overall cost and time-to-market are significantly reduced. However, hardware security becomes more subject to various kinds of tampering in the supply chain ~\cite{tehranipoor2011trustworthy}\cite{karri2010trustworthy}. Typically, a hardware system does no more than its requirements. Doing more than required, hardware Trojan horses (HTHs) can be implanted to facilitate the leakage of confidential information or cause the failure of a system~\cite{adee2008hunt}-\cite{bhunia2014hardware}. Outsourcing (e.g., third-party IP components, design tools and off-shore IC manufacturing) makes malicious HTH attacks possible. In order to introduce the motivation of deploying HTH attacks, here we make some scenario from various aspects, i.e., attackers.
Manufacturers: Given a design house A with its competitor B, in order to interfere B's commercial development, A paid B's ICs manufacturer M such that M tampered B's layout, making B's product(s) malfunction earlier than expected.
Design-tool suppliers: Given a country C with its imaginary enemy D, and we assume that D's design houses utilize at least one design tool supplied from C's software corporation S. For the purpose of obstructing D's high-tech military weapon development, C forced S to embed malicious mechanisms in its software merchandise. Therefore, no matter what kinds of military equipment D produces, they are stealthily put HTH in the inner ICs.
After the insertion of HTHs, it is difficult to prove their existence since they are pervasive and inappreciable. In this sense, the proposed research provides new primitives for aforementioned hardware security threats, by exploring the feasibility of different HTH attacks and associated detection/prevention countermeasures.
Reliability Trojan is one of the main categories of HTH attacks because its behavior is progressive and is thus not trivial to be detected, or not considered malicious. Time-de- pendent dielectric breakdown (TDDB), bias temperature in- stability (BTI), and electromigration (EM) are some of the critical failure mechanisms affecting lifetime reliability. With the continuous shrinking of transistor and interconnect dimensions, the rate of such progressive wear-out failures is getting higher. In addition, due to the increasing transistor density without proportional downscaling of supply voltage, the power density and thus the operating temperature will rise significantly, which further accelerates the failure mechanisms because they are all exponentially dependent on temperature. In this work, we propose to insert reliability Trojan into a circuit which can finely controls the circuit lifetime as specified by attackers (or even designers), based on manipulating BTI-induced aging behavior in a statistical manner.