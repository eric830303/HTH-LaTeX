\section{INTRODUCTION}
Due to escalating complexity of hardware design and manufacturing~\cite{wilson2013international}, integrated circuits (ICs) are usually designed and fabricated in multiple nations worldwide. Moreover, some design tools are also supplied from different units. With the aid of third-party IP components and off-shore IC manufacturing, the overall cost and time-to-market are significantly reduced. However, hardware security becomes more subject to various kinds of tampering in the supply chain~\cite{tehranipoor2011trustworthy}\cite{karri2010trustworthy}. Typically, a hardware system does no more than its requirements. Doing more than the required, hardware Trojan horses (HTHs) can be implanted to facilitate the leakage of confidential information or cause the failure of a system~\cite{adee2008hunt}-\cite{bhunia2014hardware}. Outsourcing (e.g., third-party IP components, design tools and off-shore IC manufacturing) makes malicious HTH attacks possible. Here, we take two motivating scenarioes for introducing from HTH attacks as follows:
\begin{enumerate}
	\item Manufacturers: Given a design house A with its competitor B, in order to interfere B's commercial development, A paid B's ICs manufacturer M such that M tampered B's layout, making B's product(s) malfunction earlier than expected.
	\item Design-tool suppliers: Given a country C with its imaginary enemy D, and we assume that D's design houses utilize at least one design tool supplied from C's software corporation S. For the purpose of obstructing D's high-tech military weapon development, C forced S to embed malicious functions in its software merchandise. As a consequence, no matter what kinds of military equipment D produces, D is unaware of the fact that their designs are stealthily inserted HTHs.
\end{enumerate}
After the insertion of HTHs, it is difficult to be aware of their existence since they are pervasive and imperceptable. In this sense, the proposed research provides new primitives for aforementioned hardware security threats, by exploring the feasibility of different HTH attacks and associated detection/prevention countermeasures.

Reliability Trojan is one of the main categories of HTH attacks because its behavior is progressive and thus hard to be detected, or not considered malicious. Time-dependent dielectric breakdown (TDDB), bias temperature instability (BTI), and electromigration (EM) are some of the critical failure mechanisms affecting lifetime reliability. With the continuous shrinking of transistor and interconnect dimensions, the rate of such progressive wear-out failures is getting higher. In addition, due to the increasing transistor density without proportional downscaling of supply voltage, the power density and thus the operating temperature will rise significantly, which further accelerates the failure mechanisms because they are all exponentially dependent on temperature. 

Among various aging mechanisms, BTI is known for prevailing over other device aging phenomena, in terms of dependence on the scaling of nanometer technologies. BTI~\cite{schroder2003negative} is a MOSFET aging phenomenon that occurs when transistors are stressed under bias (positive or negative, i.e., $V_{gs} = \pm V_{dd}$) at elevated temperature. As a result of the dissociation of Si-H bonds along the Si-SiO2 interface, BTI-induced MOSFET aging manifests itself as an increase in the threshold voltage ($V_{th}$) and decrease in the drive current ($I_{ds}$)~\cite{zafar2006negative}, which in turn lengthen the propagation delays of logic gates/paths. Experiments on MOSFET aging~\cite{chakravarthi2004comprehensive} indicate that BTI effects grow exponentially with higher operating temperature and thinner gate oxide. If the thickness of gate oxide shrinks down to 4nm, the circuit performance can be degraded by as much as 15\% after 10 years of stress and lifetime will be dominated by BTI~\cite{kimizuka1999impact}. Once the performance degradation exceeds a tolerable limit, the timing specification is no longer met and then timing errors begin to occur~\cite{kumar2006analytical}\cite{wang2010impact}.

In this work, we propose to insert reliability Trojan into a design/circuit which can controls the circuit lifetime as specified by attackers (or even designers), based on manipulating BTI-induced aging behavior in a statistical manner. Studies about reliability Trojan have been proposed since last few years. Authors of \cite{shiyanovskii2010process} detail BTI and HCI effects which induce aging failures, and accelerates the effects by aggravating the most influential parameters of BTI and HCI. In~\cite{sreedhar2012reliability}, a few Trojan designs are proposed to accelerate EM, BTI and TDDB effects by stressing/modifying specific interconnects and gates. Some studies also try to control the lifetime of a circuit by counters or timers. In \cite{yang2016a2}, authors present a Trojan which controls lifetime by analog mechanism. It siphons charge from target wire and stores to a capacitor until voltage on the capacitor rises above the threshold and sets its output flip-flop to a desired value.  The work~\cite{karimi2015magic} presents an unmodified Trojan by analyzing the netlist of a circuit to identify its critical paths; then they generate patterns/instructions for stressing those paths. These patterns can be fed by external programs or embedded devices to accelerate the aging and decrease the circuit performance and lifetime. The studies~\cite{yang2016a2} and~\cite{karimi2015magic} focus on the logic blocks which highly depend on users' operational modes. However, authors of~\cite{shiyanovskii2010process} does not estimate circuit lifetime in detail and the work of~\cite{sreedhar2012reliability} has relatively high cost based on using counters to control the lifetime of circuits. To predict circuit lifetime with Trojans, a few mathematical models are proposed in~\cite{burman2012effect} to estimate circuit reliability, but it only tries on tiny circuit C17 and does not consider aging. In addition, authors of~\cite{wei2013undetectable} propose an idea using aging effects to induce a circuit into its redundant states (i.e., operational modes) and thereafter execute malicious function.

This paper proposes a method of hardware Trojan insertion to control the lifetime of a circuit based on manipulating the rate of circuit aging. We consider ($i$) the aging of both clock trees and combinational logical paths, and ($ii$) the aging correlation among critical paths. These considerations ensure the effect of our proposed Trojan to be manifested on time under all possible workloads, from user to user. More clearly, we present a methodology of deploying duty-cycle converters (DCCs) into a clock tree to accelerate/decelerate the aging of predesignated clock buffers/inverters associated with critical paths. Those infected paths will tend to cause timing violations (or failures) around the time we set, regardless of workload variations. The advantage and contribution of our work are:
\begin{itemize}
	\item \textbf{Low Cost and High Invisibility:} Aging-based reliability Trojan uses natural, time-consuming effect of a circuit. There is no unusual behavior on performance or function when chips are just manufactured so it cannot be detected by normal tests \cite{sreedhar2012reliability}. Our inserted Trojan modifies only a low fraction of the circuit and just changes the duty cycle of some sub clock trees. The modifications are marginal so it is hard to be found from the layout. Also, clock wave form be changed is invisible except we add test points into the sub clock trees which are impacted \cite{sreedhar2012reliability}.
	\item \textbf{Low Dependence on User Operation:} We ensure that the proposed Trojan can attack the circuit successfully by analyzing the correlation between aging rate sof critical paths. Then we transform the problem to a Minimum Dominating Set (MDS) problem for minimizing the cost and guarantee at least one of the critical paths would fail under any operation mode.
\end{itemize}
