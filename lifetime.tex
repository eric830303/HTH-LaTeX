
\section{LIFETIME ESTIMATION}
\label{sec:lt_estimation}
%-------------------------- Paragraph --------------------------------------------------------------------------------------
In this section, we propose a methodology to estimate the lifetime interval of attacked designs, considering the workload variations based on the aforementioned corollary, assumption, and definition of critical operational mode.

It is worth reminding that, the corollary says the union of critical operational modes of all candidate paths is equivalent to the universe of operational modes. Thus, to consider users' all operational modes (i.e., universe of operational modes), we have to consider the severe aging of each candidate path. If a candidate path undergoes severe aging, i.e., the associated critical operational modes of the path are applied on the attacked designs, we predict the lifetime of other paths using a binary-search-based method. More specifically, given a candidate path $i$, another path $j$, and time $t$, our goal is to predict $j's$ lifetime by binarily searching for convergent $t$. That is, when $t$ converges at the end of binary search, $j's$ lifetime equals the value of $t$. 

The time $t$ varies in each iteration of binary search, depending on whether timing violation occurs on path $j$ using Equation~(\ref{eq:setup}). While Equation~(\ref{eq:setup}) is used to check $j's$ setup timing constraint, $j's$ aging rate is derived by the aging correlation between $i$ and $j$ (i.e., regression equation of $i$ on $j$) and $i's$ aging rate is derived using Equation~(\ref{eq:worst}).

  
