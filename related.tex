\section{RELATED WORK}
\label{sec:related}
Studies about reliability Trojan have been proposed since last few years. \cite{shiyanovskii2010process} details BTI and HCI effects which induce aging failures, and accelerates the effects by aggravating the most influential parameters of BTI and HCI. \cite{sreedhar2012reliability} proposes a few Trojan designs which accelerate EM, BTI and TDDB effects by stressing/modifying specific interconnects and gates. Some studies also try to control the lifetime of a circuit by counters or timers. In \cite{yang2016a2}, authors present a Trojan which controls lifetime by analog mechanism. It siphons charge from target wire and stores to a capacitor until voltage on the capacitor rises above the threshold and sets its output flip- flop to a desired value.  \cite{karimi2015magic} presents an unmodified Trojan by analyzing the netlist of a circuit to identify its critical paths; then they generate patterns/instructions for stressing those paths. These patterns can be fed by external programs or embedded devices to accelerate the aging and decrease the circuit performance and lifetime. \cite{yang2016a2} \cite{karimi2015magic} focus on the logic blocks which highly depend on users' operational modes. However, \cite{shiyanovskii2010process} does not estimate circuit lifetime in detail and \cite{sreedhar2012reliability} has relatively high cost based on using counters to control lifetime. To predict circuit lifetime with Trojans, \cite{burman2012effect} uses mathematical modeling to estimate circuit reliability, but it only tries on tiny circuit C17 and does not consider aging. In addition, \cite{wei2013undetectable} proposes an idea using aging effects to induce a circuit into its redundant states (i.e., operational modes) and thereafter execute malicious function.
This paper proposes a method of hardware Trojan insertion to control the lifetime of a circuit based on manipulating the rate of circuit aging. We consider (i) the aging of both clock trees and combinational logical paths, and (ii) the correlation of aging rates between critical paths. These considerations ensure the effect of our proposed Trojan to be manifested on time under all possible workloads due to various users' operational conditions. More clearly, we present a methodology that deploys duty-cycle converters (DCCs) into a clock tree to accelerate the aging of predesignated clock buffers/inverters associated with critical paths. Those paths will fail around the time we set regardless of operational conditions.

