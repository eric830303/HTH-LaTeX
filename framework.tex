\section{DEFINITION AND DCC INSERTION FLOW}
\label{sec:DEF}
\begin{figure}
	\centering
	\includegraphics[width=0.9\columnwidth]{flow.png}
	\caption{DCC insertion flow}
	\label{fig:flow}
\end{figure}
\begin{figure}
	\centering
	\includegraphics[width=0.9\columnwidth]{pathset.png}
	\caption{Classification of critical paths}
	\label{fig:set}
\end{figure}

Our proposed flow of DCC insertion is shown in Figure~\ref{fig:flow}. First, we read the data (e.g. circuit netlist, timing report, aging prediction parameters) and perform preprocessing. The detail about preprocessing will be described later. Then we construct a graph according to the correlation and regression between critical paths. In the graph, a vertex represents a critical path, and an edge between two vertices (i.e., a pair of critical paths) represents the correlation/regression between them. Afterwards, we choose some paths as our targets and transform our problem of DCC insertion based on SAT (Boolean Satisfiability), which can be solved by existing SAT solvers such as miniSat. Finally, we estimate the circuit lifetime after DCC insertion to evaluate the quality of our HTH attack.

%------- Definition -------------------------------------------------------------------------------------------------------------------------------
\textbf{\uline{Definition}}
\begin{itemize}
	\item $n$: Expected circuit lifetime under the proposed HTH attack based on DCC insertion. The unit is year.
	\item $\varepsilon$: Maximum tolerable error. The unit is year.
	\item \textit{Desired lifetime interval}: A desired interval of path/circuit lifetime under workload variations and statistical considerations of aging behavior among critical paths. Given $n$ and $\varepsilon$, the desired lifetime interval is $[ n - \varepsilon, n + \varepsilon]$.
\end{itemize}

%------- Terminology -----------------------------------------------------------------------------------------------------------------------------
\textbf{\uline{Terminology}}
\begin{itemize}
	\item \textit{Premature failure} denotes that a path leads to timing violations earlier than $n - \varepsilon$ years.
	\item \textit{Postmature failure} denotes that a path leads to timing violations later than $n + \varepsilon$ years.
	\item \textit{Proper failure} denotes that a path leads to timing violations within desired lifetime interval.
\end{itemize}

%------- Classification of DCC Placements --------------------------------------------------------------------------------------------------
\textbf{\uline{Classification of DCC Placements}}
\begin{itemize}
	\item \textit{Prematurely-failing DCC placement}: A kind of DCC placement which leads the critical path to premature failure.
	\item \textit{Properly-failing DCC placement}: A kind of DCC placement which leads the critical path to proper failure.
	\item \textit{Posmaturely-failing DCC placement}: A kind of DCC placement which leads the critical path to postmature failure.
	\item \textit{Illegal DCC placement (insertion)}: At most one DCC is allowed along a clock path from the clock source to a flip-flop. Placing more than one DCC along the same clock path is called illegal DCC placement.
\end{itemize}

%------- Classification of Critical Paths --------------------------------------------------------------------------------------------------
\textbf{\uline{Classification of Critical Paths}}
\begin{itemize}
	\item \textit{Candidate}: A critical path is defined as a candidate if there exists one properly-failing DCC placement.
	\item \textit{Shortlist}: A shortlist is a subset of candidates which are selected as target paths to be attacked. Attacking such paths involves placing DCCs on their associated clock paths.
	\item \textit{Mine}: A path is called a victim if one or more DCCs exist on its associated clock paths, as a result of attack on \textit{shortlist}.
	\item \textit{Effective victim}: A path is called an effective victim if it is a victim and the existing DCC(s) will lead the path to \textit{proper failure}.
\end{itemize}

Figure~\ref{fig:set} illustrates the relationship among those classification of critical paths. First, there is no intersection between candidate and mine. Second, the intersection between candidate and victim is effective victim. Third, shortlist is a subset of candidate; effective victim is a superset of shortlist, and so is victim. Moreover, victim and mine are likely to intersect because some paths in victim may belong to mine, and we have to ensure that these paths will not fail prematurely.

%------- PV-Aware Methodology --------------------------------------------------------------------------------------------------
\subsection{Classification of DCC Placements Considering Process Variations}
\label{sec:dcc_pv}
\begin{figure}
	\centering
	\includegraphics[width=0.9\columnwidth]{pvaware.png}
	\caption{Classification DCC placements considering process variation}
	\label{fig:pvaware}
\end{figure}

Note that the classification of DCC placements introduced earlier is deterministic, i.e., unaware of process-induced timing variability when referring to timing violations. In this subsection, we include the influence of process variation (PV) when classifying DCC placements, such that the resulting attack is PV-aware.
The PV-aware method is depicted in Figure~\ref{fig:pvaware}(a). Given a DCC placement on the clock path associated with one critical path, we classify the DCC placement based on Monte-Carlo simulation with the following two procedures: (\textit{i}) Impose extra positive/negative V\textsubscript{th} offsets on the transistors along the critical path. Note that, the V\textsubscript{th} offsets follow a normal distribution, with the standard deviation of a specified value, which is usually set to 30mV. That is, 68\% of V\textsubscript{th} offsets of transistors reside between positive and negative 30mV. (\textit{ii}) Then, a model, introduced in Section VI, is used to simulate the aging behavior of the critical path, such that we can estimate when the timing violation will occur. Note that, the model not only converts the V\textsubscript{th} offset to the delay shift of the transistor, but also considers the correlation of PV and BTI. The model will be explained in Section VI. Then, as illustrated in Fig. 6(a), the steps of (\textit{i}) and (\textit{ii}) are iterated for T times. Each iteration results in a time point, at which the critical path fails (i.e., timing violation occurs). Then, as shown in Figure~\ref{fig:pvaware}(b), the time points are categorized into 3 groups: premature group, proper group, and postmature group. Finally, we count the size of each group (i.e., the number of time points in the group). If the size of proper group is larger than a threshold, said H, the placement is classified as a properly-failing DCC placement. Otherwise, the sizes of the other two groups are compared. The larger one determines the classification of the DCC placement. For instance, if the size of
proper group is smaller than the threshold H, the sizes of premature group and postmature group are compared. Without the loss of generality, assume the size of postmature group is larger; then, the DCC placement is finally classified as postmaturely-failing one.

%------- Shortlist selection --------------------------------------------------------------------------------------------------
\subsection{Selection of Target Paths (Shortlist) to be Attacked}
\label{sec:shortlist}
\begin{figure}
	\centering
	\includegraphics[width=0.9\columnwidth]{graph.png}
	\caption{Example of graph used in choosing targets}
	\label{fig:graph}
\end{figure}

If an attacked critical path always ages as estimated (i.e., under worst-case aging), a successful attack can be obtained just by inserting DCCs on its associated clock paths. How- ever, uncertainty of user-dependent operational modes (e.g., watching video, playing games) can influence/vary the aging behavior. Therefore, we must ensure that the attack will succeed under any operational mode. We make the following assumption, which will be used in Section [Undefined]:

\textit{"Every operational mode causes at least one candidate path to undergo worst-case aging."}

An operational mode, which causes one path to undergo worst-case aging, is called ?critical operational mode? of the path. Further, the union of critical operational modes of all candidate paths is equivalent to the universe of operational modes. Based on the assumption, attacking all candidates is a na�ve method to guarantee successful attack. Nevertheless, it is very costly and may be impossible.

Therefore, after observing the relationship among aging of paths, we find that the aging behaviors of many paths are highly correlated. If several paths are highly correlated in terms of aging behavior, one operational mode can lead all of them to age to a similar extent. Thus, we can simply attack one out of those highly correlated paths to cover multiple operational modes. For example, given two critical paths $A$ and $B$. Their critical operational modes are $O_{A}$ and $O_{B}$, respectively. Assume that $A$ and $B$ are highly correlated in terms of aging behavior. Because $A$ and $B$ age similarly/closely, $O_{A}$ causes $A$ to age in the worst-case and also causes $B$ to age severely. Consequently, even if we simply attack path $A$, not only $O_{A}$ but also $O_{B}$ can make the attack successful (premature failure of path $A$). This property helps reduce the number of targeted paths to be considered/formulated.
To choose the attack targets (shortlist), we transform the relationship of paths to a directed graph (also known as digraph). In Figure~\ref{fig:graph}, vertices represent candidates and arcs (i.e., directed edges) are correlation coefficients ($R^2$) and linear regression equations between each pair of vertices. Each arc has a regression equation, whose coefficients are obtained by running functional simulation. $X_{i}$ denotes the worst-case aging rate of path $i$, whose exact value will be introduced in Section [Undefined]. $Y_{j}$ denotes the aging rate of path $j$ predicted based on the linear regression equation. Consider the orange equation in Figure~\ref{fig:graph}:

\begin{equation*}
	\centering
	Y_{1} = 0.98 \cdot X_{2} - 0.02
\end{equation*}

Given the worst-case aging rate of vertex/path 2, $X_{2}$, the aging rate of vertex/path 1, $Y_{1}$, can be predicted as 0.98 multiplied by $X_{2}$ minus 0.02
Before the shortlist is determined by selecting a subset of candidates in the graph, we can simplify the graph by removing some arcs which indicate the relationships of weak aging correlation between pairs of paths.

The cost of our proposed HTHs is the count of inserted DCCs. In order to minimize the cost, we must select minimum-sized targets to cover all candidate paths, that is, to dominate all candidate paths in the digraph. This problem is similar to a classical digraph problem, \textbf{Minimum Dominating Set (MDS)}:

\textit{On digraph $G = (V, E)$, find a minimum-sized set of vertices $S \subseteq V$ such that $\forall y \notin S$, $\exists x \in S$, there exists an arc from $x$ to $y$. And we say that $y$ is dominated by $x$}.


%------- SAT  --------------------------------------------------------------------------------------------------
\subsection{Generation of SAT Formula}
\label{sec:SAT}
\begin{figure}
	\centering
	\includegraphics[width=0.9\columnwidth]{SAT.png}
	\caption{Introduction of SAT-formulation}
	\label{fig:sat}
\end{figure}
\begin{figure}
	\centering
	\includegraphics[width=0.9\columnwidth]{SAT_grp.png}
	\caption{SAT-formulation and relationship among 3 groups of DCC insertions}
	\label{fig:sat_grp}
\end{figure}


After the shortlist is determined, the problem is transformed into \textbf{Conjunctive Normal Form (CNF)}. That is, it is formulated as a \textbf{Boolean satisfiability (SAT)} problem. First, as shown in Figure~\ref{fig:sat}, each clock buffer is labeled by two Boolean variables to represent four types of DCCs (none, 20\%, 40\%, 80\%) inserted ahead of it. Next, the proposed formulation is based on the classification of critical paths:
\begin{enumerate}
	\item Paths in the shortlist (i.e., targets): On their associated clock paths, formulate all conditions of prematurely-failing DCC placements and postmaturely-failing DCC placements.
	\item Other paths (paths not in the shortlist): On their associated clock paths, formulate all conditions of prematurely-failing DCC placements.
	\item All paths: Formulate all conditions of illegal DCC placements.
\end{enumerate}

Consider the example in Figure~\ref{fig:sat}: assume that critical path P is in the shortlist. If the insertion of an 80\% DCC ahead of buffer B will lead the path to premature failure, then the following clause will be added into CNF:
\begin{gather*}
	\mbox{($A_{1} \lor A_{0} \lor \neg B_{1} \lor B_{0} \lor C_{1} \lor C_{0}$) } 
\end{gather*}

Then, if the insertion of a 20\% DCC ahead of buffer C will lead the path to postmature failure, then the following clause will be added into CNF:
\begin{gather*}
	\mbox{($A_{1} \lor A_{0} \lor B_{1} \lor B_{0} \lor C_{1} \lor \neg C_{0}$) } 
\end{gather*}

More explicitly, according to the above statements, we must avoid those conditions denoted by the red shape illustrated in Figure~\ref{fig:sat_grp}, and expect to find a feasible solution denoted by the green shape. After the formulation, our proposed problem based on DCC insertion is transformed into a CNF file. The CNF can be fed to a SAT solver and we can find the locations and types of inserted DCCs by decoding the output from the solver.



