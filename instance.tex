\section{Monte-Carlo Instantiation of attacked designs}
\label{sec:ins}
After the locations of DCC insertions are determined in the existing clock tree, Monte-Carlo simulation is performed to demonstrate the influence of process variations (PVs) on the proposed Trojan. The section is organized as follows: Section~\ref{sec:ins:mc_ins} details the procedures of instantiation of attacked designs. Section~\ref{sec:ins:lt} details the lifetime estimation of Monte-Carlo instances of attacked designs, considering the correlation between PVs and aging. The correlation is discussed in Section~\ref{sec:ins:cor}. The experimental results will be shown in the next section.

\subsection{Monte-Carlo Instantiation of the Attacked Designs}
\label{sec:ins:mc_ins}
Given an attacked design, it is instantiated considering PV by imposing extra V\textsubscript{th} offset (i.e., $\Delta V\textsubscript{th}$ ) on each transistor. Note that, these offsets follow a normal distribution with the standard deviation of a given value, which usually ranges from 10mV to 30mV \cite{han2011statistical}\cite{schlunder2017influence}. In other words, if the standard deviation is set to 20mV, it implies that 68\% of V\textsubscript{th} offsets reside between $\pm$ 20mV. Up to present, a Monte-Carlo instance of an attacked design is built.
Various instances of the attacked design are generated. Each instance (i.e., Monte-Carlo seed) can be considered as a die after the circuit is manufactured. In our experiment, each attacked design is instantiated for 1000 times with a specified standard deviation of V\textsubscript{th}. %Whenever an instance is generated, we use aforementioned algorithms to estimate its lifetime interval, which consists of a lower bound and an upper bound.

\subsection{Lifetime Estimation Considering the Effect of PVs on aging rates of transistors}
\label{sec:ins:lt}
Whenever an instance is generated, Algorithm~\ref{LT_1} and~\ref{LT_2} are applied to estimate its lifetime interval. Note that, because PVs is considered, the aging rates of transistors along a path will no longer be equal; in other words, because threshold voltages of transistors along the same path are not fixed due to PVs, their aging rates differ. Thus, at line 8 in Algorithm~\ref{LT_2}, aging rate of path $i$ ($X_{i}$) must consider the aging rate of individual transistor, instead of using the deterministic Equation (\ref{eq:worst}). The aging rate of individual transistor can be obtained using the mathematical model introduced in the next subsection, which discusses the aging rate estimation of transistors, considering the correlation between PVs and BTI.%It can be modified by following procedures: ($i$) Obtain the aging latency of a path by accumulating gate delays using the above mechanism in part B. ($ii$) Then, aging rate of the path equals the aging latency divided by non-aging/fresh latency.

\subsection{Aging rate estimation of transistors considering the correlation between PVs and BTI}
\label{sec:ins:cor}
\subsubsection{Impact of PVs on BTI}
\label{sec:correlation}
In this section, we discuss the influence of process variation (PV) on BTI. Other works \cite{kiamehr2016impact}\cite{chen2013novel} do consider the PV effect while performing Monte-Carlo simulations, but ignore the correlation \cite{kiamehr2016impact} between PV and BTI. The correlation is a long-term phenomenon that bridge the V\textsubscript{th} differences among the transistors over a period. Further, a positive/negative V\textsubscript{th} offset leads to a higher/lower fresh V\textsubscript{th}, causing a lower/higher aging speed. Therefore, the gap between high and low V\textsubscript{th} will be gradually converged, letting threshold voltages of transistors, whose fresh ones are different, reach a convergent value.

\subsubsection{Model of the Correlation}
A model in~\cite{gomez2016early} is proposed to estimate the correlation between fresh V\textsubscript{th} offset and BTI effects:
\begin{equation}
	\centering
	\fontsize{9}{9} \selectfont
	\Delta V_{th\_nbti} = (1 - S_{v} \cdot \Delta V_{th\_pv})  \cdot A \cdot a^n \cdot t^n
	\label{eq:cor}
\end{equation}
\begin{equation}
	\centering
	\fontsize{9}{9} \selectfont
	V_{th} = \Delta V_{th\_nbti} + \Delta V_{th\_pv} + V_{th\_design}
	\label{eq:conv}
\end{equation}
$\Delta V_{th\_pv}$ denotes the fresh V\textsubscript{th} due to PV. $\Delta V_{th\_nbti}$ denotes the BTI-induced V\textsubscript{th} shift and $V_{th\_design}$ denotes the nominal threshold voltage of the design. $S_{v}$ depends on $\Delta V_{th\_pv}$, and can be derived by following procedures:

\paragraph{Assume the value of $V_{th}$ is convergent}
We assume threshold voltages of all transistors will be convergent after a long period, even if their fresh values are different. In other words, $V_{th}$ is fixed regardless of various $\Delta V_{th\_pv}$, since the aforementioned correlation takes effect. 

\paragraph{Obtain the convergent value of $V_{th}$}
Since $V_{th}$ is fixed regardless of various $\Delta V_{th\_pv}$, we set $\Delta V_{th\_pv}$ to 0 in Equation~(\ref{eq:conv}) to derive the convergent value of $V_{th}$. This way, Equation~(\ref{eq:conv}) can be simplified as Equation~(\ref{eq:conv2}), where the convergent value of $V_{th}$ equals $\Delta V_{th\_nbti}$ plus $V_{th\_design}$. Here, $V_{th\_design}$ is given by technology and $\Delta V_{th\_nbti}$ can be simplified as Equation~(\ref{eq:cor2}) because $V_{th\_pv}$ is set to 0. In Equation~(\ref{eq:conv2}), since $V_{th\_design}$ is known and $\Delta V_{th\_nbti}$ can be derived without unknown $S_{v}$, the convergent value of $V_{th}$ can be obtained.
\begin{equation}
	\centering
	\fontsize{9}{9} \selectfont
	\Delta V_{th\_nbti} = A \cdot a^n \cdot t^n
	\label{eq:cor2}
\end{equation}
\begin{equation}
	\centering
	\fontsize{9}{9} \selectfont
	V_{th} = \Delta V_{th\_nbti} + 0 + V_{th\_design}
	\label{eq:conv2}
\end{equation}
\paragraph{Obtain the value of $S_{v}$ with specific $\Delta V_{th\_pv}$}
Given a specific value of $\Delta V_{th\_pv}$, our objective is to derive corresponding $S_{v}$ value. Since the convergent $V_{th}$ value is obtained in the last step and $V_{th\_design}$ is known, we can derive the corresponding $\Delta V_{th\_nbti}$ using Equation~(\ref{eq:conv}), such that the corresponding $S_{v}$ can be obtained in Equation~(\ref{eq:cor}). 

%Shortly speaking, given a specific value of $\Delta V_{th\_pv}$, our objective is to derive corresponding $S_{v}$ value. We first assume threshold voltages are convergent over a long-term period; then, since all the coefficients of Equation~(\ref{eq:cor}) are known, the value of $\Delta V_{th\_nbti}$ can be obtained.
So far, the conversion from a given specific $\Delta V_{th\_pv}$ to corresponding $\Delta V_{th\_nbti}$ has been constructed. Then, $\Delta V_{th\_nbti}$ must be transformed to aging-induced delay shift. In~\cite{wang2007efficient}, the delay shift is linearly proportional to $\Delta V_{th\_nbti}$:
\begin{equation}
	\centering
	\fontsize{9}{9} \selectfont
	\Delta t_{p\_aged} = C \cdot \Delta V_{th\_nbti}
	\label{eq:vtodelay}
\end{equation}	
where $\Delta t_{p\_aged}$ is BTI-induced delay shift, and $C$ is a constant and fitted to 0.5 after SPICE simulation. Further, the Equation~(\ref{eq:vtodelay}) is modified as following Equation~(\ref{eq:vtodelay2}) to account for the conversion from $\Delta V_{th\_pv}$ to intrinsic delay shift. 
\begin{equation}
	\centering
	\fontsize{9}{9} \selectfont
	\Delta t_{p\_intrinsic} = C \cdot \Delta V_{th\_pv}
	\label{eq:vtodelay2}
\end{equation}	
where $\Delta t_{p\_intrinsic}$ is the delay shift caused by $\Delta V_{th\_pv}$. Up to now, a model is built to convert a given specific $\Delta V_{th\_pv}$ to corresponding $\Delta t_{p\_aged}$ and $\Delta t_{p\_intrinsic}$. The model is involved in Section[] and Section[]. 
%\subsection{Lifetime Distribution of Monte-Carlo Instances}
%\label{sec:exp:exp}
%Figure~\ref{fig:exp1}, Figure~\ref{fig:exp2}, and Figure~\ref{fig:exp3} show the lifetime distributions of instances of the attacked three designs (s38417, des\_perf and leo3mp). The designs are attacked to fail at 3rd, 4th or 5th year. Note that, there exists no SAT solution while leo3mp is attacked to fail at 3rd year, whereas there exists SAT solution while it is attacked to fail at 4th year. In each left/right subfigure, there exist four distributions. The distributions of various colors differ in the standard deviations of V\textsubscript{th} while generating instances. That is, each color corresponds to one distinct value of standard deviation of V\textsubscript{th}. In our experiments, the deviations are set to 15mV, 20mV, 25 mV and 30 mV, respectively. Moreover, in each distribution, there exist two peaks. The left/right peak denotes the distribution of lower/upper bounds of lifetime intervals of in- stances. Note that, there exist two differences between Figure~\ref{fig:benchmark} and the three figures (Figure~\ref{fig:exp1}, Figure~\ref{fig:exp2}). First, the designs in Figure~\ref{fig:benchmark} are Trojan-free ones, instead of Trojan-included ones in the three figures. Second, because the lifetime intervals of the Trojan-free designs are not subject to the process variation, the original lifetime intervals (5th column in Figure~\ref{fig:benchmark}) do not consider the effect of PV.
%Apparently, Figure~\ref{fig:exp1}, Figure~\ref{fig:exp2}, and Figure~\ref{fig:exp3} show that, as the standard deviation becomes larger, the interval between the left and right peaks becomes wider; that is, the larger standard deviation leads to a less accurate attack. Therefore, the lifetime accuracy of the proposed Trojan is impacted by the diversity of threshold voltages. However, even though the peaks of two bounds deviate from the desired lifetime interval, it does not mean that the attacked designs must not fail in that interval. As mentioned in Section V, the estimated lifetime interval of one instance consists of two bounds. One is lower bound; and the other is upper bound. The two bounds denote the earliest and the last time points, at which the instance will fail. But the exact time point, at which the instance fails, depends on the workload. Therefore, since the lifetime interval of each instance is overlapped with the de- sired lifetime interval, the proposed Trojan is still likely to control the design lifetime in that interval.

%\subsection{Detectability}
%When it comes to the detectability of the proposed Trojan, side-channel analysis is often used to detect the existence of hardware Trojan. Nevertheless, the used DCC count is marginal compared with the total gate count. On average, DCC count is less than 0.2\% of total gate count. That is, the area overhead is insignificant. Also, the power overhead due to DCCs can be regarded as the power variations caused by PV. Therefore, the proposed Trojan framework is difficult to be detected by conventional side-channel analysis.
%Some Trojan defenders can insert probes in the clock network to inspect the variation of clock duty cycle. The detection method is indeed able to prove the existence of the proposed Trojan. However, the method need to be supported by extra I/O pins/ports. Therefore, it is impractical, not only because the pin counts of ICs are limited, but also the area over- head of extra pins is costly.

%\section{CONCLUSION}
%We proposed a methodology of hardware Trojan insertion to control the circuit lifetime with the consideration of aging behavior, correlation between pairs of critical paths and process variations. The influence of Trojan heavily reduce the lifetime of circuit instances. Even though the accuracy is impacted by PV, the lifetime of instances is still likely to fail within the desired lifetime interval $[0.9 \times n, 1.1 \times n]$. Also, the DCC count is less than 0.2\% of total gate count, implying limited area and power overhead. Therefore, the proposed Trojan is difficult to be detected.
