\section{Monte-Carlo Instantiation of attacked designs}
\label{sec:ins}
After the locations of DCC insertions are determined in the existing clock tree, Monte-Carlo simulation is performed to demonstrate the influence of process variations (PVs) on the proposed Trojan. The section is organized as follows: Section~\ref{sec:ins:mc_ins} details the procedures of instantiation of attacked designs. Section~\ref{sec:ins:lt} details the lifetime estimation of Monte-Carlo instances of attacked designs, considering the correlation between PVs and aging. The correlation is discussed in Section~\ref{sec:ins:cor}. The experimental results will be shown in the next section.

\subsection{Monte-Carlo Instantiation of the Attacked Designs}
\label{sec:ins:mc_ins}
%Given an attacked design, it is instantiated considering PV by imposing extra V\textsubscript{th} offset (i.e., $\Delta V\textsubscript{th}$ ) on each transistor. Note that, these offsets follow a normal distribution with the standard deviation of a given value, which usually ranges from 10mV to 30mV \cite{han2011statistical}\cite{schlunder2017influence}. In other words, if the standard deviation is set to 20mV, it implies that 68\% of V\textsubscript{th} offsets reside between $\pm$ 20mV. Up to present, a Monte-Carlo instance of an attacked design is built. Various instances of the attacked design are generated. Each instance (i.e., Monte-Carlo seed) can be considered as a die after the circuit is manufactured. In our experiment, each attacked design is instantiated for 1000 times with a specified standard deviation of V\textsubscript{th}. 
Given an attacked design, each Monte-Carlo instance of the design is generated by imposing extra V\textsubscript{th} offset (i.e., $\Delta V\textsubscript{th}$ ) on each transistor. The offsets follow a normal distribution with the standard deviation of a given value (usually 10mV - 30mV~\cite{han2011statistical}\cite{schlunder2017influence}). Each instance (i.e., Monte-Carlo seed) can be considered as a die after the circuit is manufactured. In our experiment, each attacked design is instantiated for 1000 times with a specified standard deviation of V\textsubscript{th}. 

\subsection{Lifetime Estimation Considering the Effect of PVs on aging rates of transistors}
\label{sec:ins:lt}
Whenever an instance is generated, Algorithm~\ref{LT_1} and~\ref{LT_2} are applied to estimate its lifetime interval. Note that, because threshold voltages of transistors along the same path are not fixed due to PVs, their aging rates differ. Thus, at line 8 in Algorithm~\ref{LT_2}, aging rate of path $i$ ($X_{i}$) must consider the aging rate of individual transistor, instead of using the deterministic Equation (\ref{eq:worst}). The aging rate of individual transistor can be obtained using the mathematical model introduced in the next subsection, which discusses the aging rate estimation of transistors, considering the correlation between PVs and BTI.%It can be modified by following procedures: ($i$) Obtain the aging latency of a path by accumulating gate delays using the above mechanism in part B. ($ii$) Then, aging rate of the path equals the aging latency divided by non-aging/fresh latency.

\subsection{Aging rate estimation of transistors considering the correlation between PVs and BTI}
\label{sec:ins:cor}
%\subsubsection{Impact of PVs on BTI}
%\label{sec:correlation}
%In this subsection, we discuss the influence of correlation of PVs on BTI. Other works \cite{kiamehr2016impact}\cite{chen2013novel} do consider the PVs effect while performing Monte-Carlo simulations, but ignore the correlation \cite{kiamehr2016impact} between PV and BTI. The correlation is a long-term phenomenon that bridge the V\textsubscript{th} differences among the transistors over a period. Further, a positive/negative V\textsubscript{th} offset leads to a higher/lower fresh V\textsubscript{th}, causing a lower/higher aging speed. Therefore, the gap between high and low V\textsubscript{th} will be gradually converged, letting threshold voltages of transistors, whose fresh ones are different, reach a convergent value.
In this subsection, we discuss the correlation between PVs and BTI. The correlation is a long-term phenomenon that bridge the V\textsubscript{th} differences among the transistors over a period. Further, a positive/negative V\textsubscript{th} offset leads to a higher/lower fresh V\textsubscript{th}, causing a lower/higher aging speed. Therefore, the gap between high and low V\textsubscript{th} will be gradually converged, letting threshold voltages of transistors, whose fresh ones are different, reach a convergent value.

%\subsubsection{Model of the Correlation}
A model in~\cite{gomez2016early} is proposed to estimate the correlation between fresh V\textsubscript{th} offset and BTI effects:
\begin{equation}
	\centering
	\fontsize{9}{9} \selectfont
	\Delta V_{th\_nbti} = (1 - S_{v} \cdot \Delta V_{th\_pv})  \cdot A \cdot a^n \cdot t^n
	\label{eq:cor}
\end{equation}
\begin{equation}
	\centering
	\fontsize{9}{9} \selectfont
	V_{th} = \Delta V_{th\_nbti} + \Delta V_{th\_pv} + V_{th\_design}
	\label{eq:conv}
\end{equation}
$\Delta V_{th\_pv}$ denotes the fresh V\textsubscript{th} offset due to PV. $\Delta V_{th\_nbti}$ denotes the BTI-induced V\textsubscript{th} shift and $V_{th\_design}$ denotes the nominal threshold voltage of the design. $S_{v}$ depends on $\Delta V_{th\_pv}$, and can be derived by following procedures:

\paragraph{Assume the value of $V_{th}$ is convergent}
We assume threshold voltages of all transistors will be convergent after a long period, even if their fresh values are different. In other words, $V_{th}$ is fixed regardless of various $\Delta V_{th\_pv}$, since the aforementioned correlation takes effect. 

\paragraph{Obtain the convergent value of $V_{th}$}
%Since $V_{th}$ is fixed regardless of various $\Delta V_{th\_pv}$, we set $\Delta V_{th\_pv}$ to 0 in Equation~(\ref{eq:conv}) to derive the convergent value of $V_{th}$. This way, Equation~(\ref{eq:conv}) can be simplified as Equation~(\ref{eq:conv2}), where the convergent value of $V_{th}$ equals $\Delta V_{th\_nbti}$ plus $V_{th\_design}$. Here, $V_{th\_design}$ is given by technology and $\Delta V_{th\_nbti}$ can be simplified as Equation~(\ref{eq:cor2}) because $V_{th\_pv}$ is set to 0. In Equation~(\ref{eq:conv2}), since $V_{th\_design}$ is known and $\Delta V_{th\_nbti}$ can be derived without unknown $S_{v}$, the convergent value of $V_{th}$ can be obtained.
Since $V_{th}$ is fixed regardless of various $\Delta V_{th\_pv}$, we set $\Delta V_{th\_pv}$ to 0 in Equation~(\ref{eq:conv}) to derive the convergent value of $V_{th}$. This way, the convergent value of $V_{th}$ equals $\Delta V_{th\_nbti}$ plus $V_{th\_design}$. Here, $V_{th\_design}$ is given by technology. In Equation~(\ref{eq:conv}), since $V_{th\_design}$ is known and $\Delta V_{th\_nbti}$ can be derived without unknown $S_{v}$, the convergent value of $V_{th}$ can be obtained.
\begin{comment}
	\begin{equation}
		\centering
		\fontsize{9}{9} \selectfont
		\Delta V_{th\_nbti} = A \cdot a^n \cdot t^n
		\label{eq:cor2}
	\end{equation}
	\begin{equation}
		\centering
		\fontsize{9}{9} \selectfont
		V_{th} = \Delta V_{th\_nbti} + 0 + V_{th\_design}
		\label{eq:conv2}
	\end{equation}
\end{comment}
\paragraph{Obtain the value of $S_{v}$ with specific $\Delta V_{th\_pv}$}
Given a specific value of $\Delta V_{th\_pv}$, our objective is to derive corresponding $S_{v}$ value. Since the convergent $V_{th}$ value is obtained in the last step and $V_{th\_design}$ is known, we can derive the corresponding $\Delta V_{th\_nbti}$ using Equation~(\ref{eq:conv}), such that the corresponding $S_{v}$ can be obtained in Equation~(\ref{eq:cor}). 

So far, the conversion from a given specific $\Delta V_{th\_pv}$ to corresponding $\Delta V_{th\_nbti}$ has been constructed. Then, $\Delta V_{th\_nbti}$ must be transformed to aging-induced delay shift. In~\cite{wang2007efficient}, the delay shift is linearly proportional to $\Delta V_{th\_nbti}$:
\begin{equation}
	\centering
	\fontsize{9}{9} \selectfont
	\Delta t_{p\_aged} = C \cdot \Delta V_{th\_nbti}
	\label{eq:vtodelay}
\end{equation}	
where $\Delta t_{p\_aged}$ is BTI-induced delay shift, and $C$ is a constant and fitted to 0.5 after SPICE simulation. Further, the Equation~(\ref{eq:vtodelay}) is modified as following Equation~(\ref{eq:vtodelay2}) to account for the conversion from $\Delta V_{th\_pv}$ to intrinsic delay shift. 
\begin{equation}
	\centering
	\fontsize{9}{9} \selectfont
	\Delta t_{p\_intrinsic} = C \cdot \Delta V_{th\_pv}
	\label{eq:vtodelay2}
\end{equation}	
where $\Delta t_{p\_intrinsic}$ is the delay shift caused by $\Delta V_{th\_pv}$. Up to now, a model is built to convert a given specific $\Delta V_{th\_pv}$ to corresponding $\Delta t_{p\_aged}$ and $\Delta t_{p\_intrinsic}$. The model is involved in Section~\ref{sec:lt_estimation} while estimating the lifetime interval of Monte-Carlo instances of designs. 

