\section{EXPERIMENTAL SETTING AND RESULTS}
\label{sec:mot}

\begin{table*}
\centering
\caption{Benchmark information and average probabilities of Monte-Carlo instances}
	\begin{tabular}{l}
	\includegraphics[width=1.9\columnwidth]{./experiment/benchmarkinfo.png}
	\end{tabular}
\label{table:benchmark}
\end{table*}

%%---------TEST--------------
\begin{figure*}[!ht]
    \centering
    \subfigure[\textit{s38417} is attacked with $n$ = 3 yr and $\varepsilon = 0.3$ yr]{
    	\label{fig:sub:s38417_3y}
        \includegraphics[width=0.85\columnwidth]{./experiment/s384173y.png}
    }
    %\hspace{0.1cm}
    \subfigure[\textit{s38417} is attacked with $n$ = 5 yr and $\varepsilon = 0.5$ yr]{
    	\label{fig:sub:s38417_5y}
        \includegraphics[width=0.85\columnwidth]{./experiment/s384175y.png}
    }
    %\hspace{0.1cm}
    \subfigure[\textit{des\_perf} is attacked with $n$ = 3 yr and $\varepsilon = 0.3$ yr]{
    	\label{fig:sub:des_3y}
        \includegraphics[width=0.85\columnwidth]{./experiment/des3y.png}
    }
    \subfigure[\textit{des\_perf} is attacked with $n$ = 5 yr and $\varepsilon = 0.5$ yr]{
    	\label{fig:sub:des_5y}
        \includegraphics[width=0.85\columnwidth]{./experiment/des5y.png}
    }
    \subfigure[\textit{leo3mp} is attacked with $n$ = 4 yr and $\varepsilon = 0.4$ yr]{
    	\label{fig:sub:3mp_3y}
        \includegraphics[width=0.85\columnwidth]{./experiment/3mp4y.png}
    }
    %\hspace{0.1cm}
    \subfigure[\textit{leo3mp} is attacked with $n$ = 5 yrs and $\varepsilon = 0.5$ yr]{
    	\label{fig:sub:3mp_5y}
        \includegraphics[width=0.85\columnwidth]{./experiment/3mp5y.png}
    }
    \caption{Lifetime distributions of Monte-Carlo Instances of Trojan-included \textit{s38417}, \textit{des\_perf}, and \textit{leo3mp}}
    \label{fig:exp}
\end{figure*}

\subsection{Experimental Setting}
\label{sec:exp:tc}
In our experiments, the benchmarks are picked out from IWLS'05 and ISCAS'89. The used technology is TSMC 65nm GP standard cell series. The utilized SAT solver is MiniSAT 2.2. The information of each design/benchmark is reported in Table~\ref{table:benchmark}, where the clock periods (Column 6) are specifically set to make the designs fail at the specified time (in our experiment, 7 years) in the presence of aging. The resulting clock period is both used in Trojan-free and Trojan-included (attacked) designs. 
\subsection{Monte-Carlo Instantiation of the Attacked Designs}
\label{sec:ins:mc_ins}
Given an attacked design, each Monte-Carlo instance of the design is generated by imposing extra $V_{th}$ offset (i.e., $\Delta V_{th}$ ) on each transistor. The $V_{th}$ offsets follow a normal distribution with the standard deviation ($\sigma_{V_{th}}$) of a given value (usually 10mV$\sim$25mV~\cite{han2011statistical}\cite{schlunder2017influence}). In our experiment, each attacked design is instantiated for 1000 times with a specified value of $\sigma_{V_{th}}$. Then, we estimate the lifetime interval and probability of each instance based on the method proposed in Section~\ref{sec:lt_estimation}. %Note that, because threshold voltages of transistors are not the same due to PVs, their aging rates differ with each others. As a consequence, we cannot use the deterministic Equation (\ref{eq:worst}) to derive the severe aging rates of candidate paths. Instead, we derive the aging rate of individual transistor based on the mathematical model in Section~\ref{sec:frame} regarding the correlation between PVs and BTI.

\subsection{Lifetime Distribution of Monte-Carlo Instances}
\label{sec:exp:exp}
Figure~\ref{fig:exp} shows the lifetime distributions of instances of the attacked three designs (\textit{s38417}, \textit{des\_perf} and \textit{leo3mp}). Due to the page limitation, the experimental results of other benckmarks are not shown in Figure~\ref{fig:exp}. The designs are attacked to make them fail at $3^{rd}$, $4^{th}$ or $5^{th}$ year (i.e., $n = 3, 4,$ or $5$)\footnote{There exists no SAT solution for \textit{leo3mp} while $n = 3$, and for \textit{netcard} while $n = 3$ or $4$. However, there exists solution for \textit{leo3mp} while $n = 4$.}. In each subfigure, there exist three distributions, each of distributions corresponds to one standard deviation of $V_{th}$ ($\sigma_{V_{th}}$) while generating Monte-Carlo instances. In our experiments, the $\sigma_{V_{th}}$ is set to 15mV, 20mV, and 25mV, respectively. As we can observe there are two peaks in each distribution. The left/right peak denotes the distribution of lower/upper bounds of lifetime intervals of instances. %Note that, the instances in Figure~\ref{fig:exp} are instantiated from Trojan-included designs, instead of Trojan-free counterparts, whose lifetimes are not subject to PVs. Thus, we do not generate the Monte-Carlo instances for Trojan-free designs .
As shown in Figure~\ref{fig:exp} apparently, as $\sigma_{V_{th}}$ becomes larger, the interval between the left and right peaks becomes wider, indicating the larger $\sigma_{V_{th}}$ leads to fewer accurate attacks. Namely, the lifetime accuracy of the proposed Trojan is impacted by the $V_{th}$ on transistors, whereas it does not mean that the attacked designs must not fail in the desired interval $[n-\varepsilon,n+\varepsilon]$. In fact, the exact fail time point of instance varies with workload.  Thus, we use the method proposed in Section~\ref{sec:lt_estimation} to estimated the probabilities that instances fail within $[n-\varepsilon,n+\varepsilon]$. The average of resulting probabilities are listed in Table~\ref{table:benchmark}, where clearly shows that the proposed Trojan attacks can successfully  control the lifetime of attacked designs within $[n-\varepsilon,n+\varepsilon]$.%Since the lifetime interval of each instance is overlapped with the desired lifetime interval $[n - \varepsilon, n + \varepsilon]$, the proposed Trojan is still likely to control the design lifetime in that interval.

\subsection{Detectability of the Proposed Trojan Attack}
\label{sec:exp:det}
A side-channel analysis is a countermeasure often taken for detecting the existence of hardware Trojan. Nevertheless, the used DCC count is marginal compared with the total gate count. On average, DCC count is less than 0.2\% of total gate count. That is, the area overhead is insignificant. In addition, the power overhead due to DCCs can be regarded as the power variations caused by PVs. As a result, the proposed Trojan is difficult to detect by the conventional side-channel analysis. In addition, some Trojan defenders can insert on-chip monitors in the clock network to monitor the variation of clock duty cycle. However, the method needs extra I/O pins/ports, which is impractical because of the limited pin counts of ICs and area overhead.


\section{CONCLUSION}
We propose a framework of hardware Trojan insertion to control the circuit lifetime, considering process variations and aging correlations between pairs of critical paths. The influence of Trojan heavily reduce the lifetime of circuit instances. Even though the accuracy is impacted by PVs, the lifetime of instances is still likely to fail within the desired lifetime interval. More important, the area overhead of inserted DCCs is less than an average of 0.2\% of total gate count, implying that the proposed Trojan is very difficult to detect and succeeds in decreasing the lifetime of attacked designs.
