 
\documentclass[10pt, conference]{IEEEtran}


\usepackage{graphicx}
\usepackage{subfigure}
\usepackage{amsmath, amsthm, amssymb}
\usepackage{multirow}
%LaTeX package conflicts with several others over the use of the algorithm identifier.  
%A common indicator is something like this message: Too many }'s.l.1616     }
%To resolve the issues, simply put the following just before the inclusion of the algorithm2
\makeatletter
\newif\if@restonecol
\makeatother
\let\algorithm\relax
\let\endalgorithm\relax
\usepackage[linesnumbered,boxed,ruled]{algorithm2e}
\usepackage{csquotes}
\usepackage[style=ieee,backend=bibtex8,maxnames=3,minnames=1,sorting=none]{biblatex}
\usepackage[version=4]{mhchem}
\usepackage[bottom]{footmisc}
\usepackage[normalem]{ulem}
\usepackage{afterpage}
\usepackage{comment}
\usepackage[normalem]{ulem}
\usepackage{algorithmic}
\usepackage{algorithm2e}
% graphics path
\graphicspath{{figures/}}
% figure name bold
\renewcommand{\figurename}{\textbf{Figure}}
% table name bold
\renewcommand{\tablename}{\textbf{Table}}
% footnote mark
\renewcommand{\thefootnote}{\fnsymbol{footnote}}
% add bib resource
\addbibresource{main.bib} 
% class format
\newtheoremstyle{mystyle}  % Name
  {}%                                     % Space above
  {}%                                     % Space below
  {}%                                     % Body font
  {}%                                     % Indent amount
  {\bfseries}%                       % Theorem head font
  {:}%                                    % Punctuation after theorem head
  { }%                                    % Space after theorem head, ' ', or \newline
  {}%                                     % Theorem head spec (can be left empty, meaning `normal')
\theoremstyle{mystyle}
\newtheorem{class}{Class}
%bibtex font size
\renewcommand*{\bibfont}{\footnotesize}

% correct bad hyphenation here
\hyphenation{op-tical net-works semi-conduc-tor}


\begin{document}

\title{Lifetime Reliability Trojan based on Exploring Malicious Aging}


% author names and affiliations
% use a multiple column layout for up to three different
% affiliations

\begin{comment}
\author{\IEEEauthorblockN{Tien-Hung Tseng, Shou-Chun Li and Kai-Chiang Wu}
	\IEEEauthorblockA{Department of Computer Science\\
	National Chiao Tung University, Hsinchu, Taiwan \\
	E-mail: \{eric830303.cs05g@g2.nctu.edu.tw, scli.cs02g@nctu.edu.tw and kcw@cs.nctu.edu.tw\}}
}
\end{comment}

% use for special paper notices
%\IEEEspecialpapernotice{(Invited Paper)}


% make the title area
\maketitle

% As a general rule, do not put math, special symbols or citations
% in the abstract
\fontsize{9}{11}\selectfont
\input abstract
\input introduction
%\input related
\input motivate
\input framework
\input lifetime
%\input instance
%\input exp_set_correlation
\input exp_other


\IEEEpeerreviewmaketitle



\printbibliography




% that's all folks
\end{document}


